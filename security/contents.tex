% Since this a solution template for a generic talk, very little can
% be said about how it should be structured. However, the talk length
% of between 15min and 45min and the theme suggest that you stick to
% the following rules:  

% - Exactly two or three sections (other than the summary).
% - At *most* three subsections per section.
% - Talk about 30s to 2min per frame. So there should be between about
%   15 and 30 frames, all told.


\section{What's security about?}

\subsection{Security and Reliability}

\begin{frame}
  \begin{block}{Reliability}
    \begin{itemize}
      \item Reliability addresses the consequences of unintentional errors.

      \item On a PC (offline), you are in control of the software components 
        sending input to each other.

      \item The aim is to avoid mistakes.
    \end{itemize}
  \end{block}

  \pause{}

  \begin{block}{Security}
    \begin{itemize}
      \item Once online, hostile adversaries can provide input.

      \item Protection against mistakes is not enough --- they will ensure to 
        make them for you.

      \item And they will do things that you can never accomplish by mistake.
    \end{itemize}
  \end{block}
\end{frame}

\begin{frame}
  \begin{itemize}
    \item To make software more reliable, it is tested against typical usage 
      patterns.

    \item To make software more secure, it has to be tested against non-typical 
      usage patterns.

    \item Testing against non-typical usage patterns might be difficult.

    \item Think before you do \dots
  \end{itemize}
\end{frame}

\subsection{Security policy}

\begin{frame}
  \begin{itemize}
    \item The main purpose of security is to protect assets.

      \pause

    \item We define our goals in a \emph{security policy}.
    \item \Eg who may access what asset and how.

      \pause

    \item We then must enforce our policy.
  \end{itemize}
\end{frame}

\begin{frame}
  \begin{example}
    \begin{itemize}
      \item Asset: private photos.

      \item Policy:
        \begin{itemize}
          \item Only I can access my photo collection.
          \item I can share individual photos to individual persons.
        \end{itemize}

        \pause

      \item Enforcement?
    \end{itemize}
  \end{example}
\end{frame}

\begin{frame}
  \begin{itemize}
    \item Each of our mechanisms is more or less \emph{trustworthy}.
    \item A trustworthy mechanism \emph{will not break} our security policy.
    \item A trusted mechanism \emph{is trusted} to not break, but actually 
      might break.
  \end{itemize}
\end{frame}

\subsection{Protection strategies}

\begin{frame}
  \begin{block}{Protection strategies}
    \begin{description}
      \item[Prevention] taking measures that prevent your assets from being 
        damaged.

      \item[Detection] taking measures that allow detection of when, how, and 
        by who an asset has been damaged.

      \item[Reaction] taking measures that allow to recover assets or recover 
        from damage to assets.
    \end{description}
  \end{block}
\end{frame}

\begin{frame}
  \begin{example}[Private property]
    \begin{description}
      \item[Prevention] Locks on doors, window bars, surrounding walls, \dots
      \item[Detection] Stolen items are missing, burglar alarms, video 
        surveillance, \dots
      \item[Reaction] Call the police, replace stolen items (insurance?), \dots
    \end{description}
  \end{example}

  \pause

  \begin{example}[Private photos]
    \begin{description}
      \item[Prevention] Encrypt, store securely.
      \item[Detection] Our photos have leaked.
      \item[Reaction] Anything other than get angry?
    \end{description}
  \end{example}
\end{frame}

\subsection{Security objectives}

\begin{frame}
  \begin{description}
    \item[Confidentiality] Concerns unauthorized disclosure of information.
    \item[Integrity] Concerns unauthorized modification.
    \item[Availability] Concerns unauthorized withholding of information or 
      resources.
  \end{description}
\end{frame}

\begin{frame}
  \begin{definition}[Data and information]
    \textcquote{hansen1973operating}[.]{%
      \textins{Data is the} \textins*{p}hysical phenomena chosen by convention 
      to represent certain aspects of of our conceptual and real world.
      \emph{The meanings we assign to data are called information.} \textins{My 
        emphasis.}
      Data is used to transmit and store information and to derive new 
      information by manipulating the data according to formal rules%
    }
  \end{definition}
\end{frame}

\begin{frame}
  \begin{example}
    \begin{itemize}
      \item \(x + 1 = 0\iff x = -1\): the two equations are data, we use the 
        formal rules of linear equations to derive the value of \(x\).
    \end{itemize}
  \end{example}

  \begin{example}
    \begin{itemize}
      \item Timestamps of TCP packets on the network with port set to 80 or 443, 
        formal rules of statistical analysis.
    \end{itemize}
  \end{example}
\end{frame}

\begin{frame}
  \begin{block}{Confidentiality}
    \begin{itemize}
      \item Unauthorized \enquote{reading}.

      \item Think about what to hide: the content of a document, or the 
        document's existence?
    \end{itemize}
  \end{block}
\end{frame}

\begin{frame}
  \begin{block}{Integrity}
    \begin{itemize}
      \item Unauthorized \enquote{writing}.

      \item Data integrity: \blockquote{The state that exists when computerized 
          data is the same as that in the source document and has not been exposed 
          to accidental or malicious alteration or destruction.}

      \item Concerns detection and correction of intentional and unintentional 
        modifications of data.
    \end{itemize}
  \end{block}
\end{frame}

\begin{frame}
  \begin{block}{Integrity continued}
    \begin{itemize}
      \item Clark and Wilson:
        \blockquote{No user of the system, even if authorized, may be permitted to 
          modify data in such a way that assets or accounting records of the 
          company are lost or corrupted.}

      \item \Ie make sure that everything is as it is supposed to.

      \item Integrity is a prerequisite to many other security services.
    \end{itemize}
  \end{block}
\end{frame}

\begin{frame}
  \begin{block}{Availability}
    \begin{itemize}
      \item This is the property of being \emph{available and usable} upon demand 
        by an authorized principal.

      \item Denial of Service (DoS) is an attack on availability which prevents 
        authorized access to resources or the delaying of time-critical 
        operations.

      \item Distributed Denial of Service (DDoS) gets much attention, this can 
        also caused by reliability problems (unintentional).
    \end{itemize}
  \end{block}
\end{frame}

\begin{frame}
  \begin{example}[DDoS]
    \begin{itemize}
      \item Attackers infect \(x\cdot 10^5\) devices (smartphones, IoT-stuff) 
        with malware.
      \item Attackers commands all devices to send requests to a given address.
      \item No one can communicate with that address under that load.
    \end{itemize}
  \end{example}
\end{frame}


\section{Principles of security}

\subsection{Fundamental design decisions}

\begin{frame}
  \begin{enumerate}
    \item Where to focus security controls?
    \item Where to place security controls?
    \item Complexity or assurance?
    \item Centralised or decentralised control?
    \item Blocking access to the layer below?
  \end{enumerate}
\end{frame}

\begin{frame}
  \begin{block}{Focus and placement of control}
    \begin{itemize}
      \item Focus of control may be on data, operations, or users.

      \item If we look at the control of integrity, its requirements may refer to 
        rules on:
        \begin{itemize}
          \item Format and content of data items, \eg account balance must be 
            integer.

          \item Operations that may be performed on a data item, \eg credit, debit 
            and transfer.

          \item Users who are allowed access to a data item, \eg account holder 
            and bank clerk.
        \end{itemize}
    \end{itemize}
  \end{block}
\end{frame}

%\begin{frame}
%  \begin{block}{Man-Machine Scale}
%    \begin{itemize}
%      \item Applications (focus on users, information)
%      \item Services (middleware)
%      \item Operating system
%      \item Operating system kernel
%      \item Hardware (focus on data, generic)
%    \end{itemize}
%  \end{block}
%\end{frame}

\begin{frame}
  \begin{block}{Complexity or Assurance}
    \begin{itemize}
      \item Generic mechanisms are simple, applications are usually feature rich.

      \item The fundamental dilemma:
        \begin{itemize}
          \item Simple generic mechanisms may not match specific security 
            requirements.

          \item To choose the right features from a rich selection, you need to 
            be a security expert.

          \item Security-unaware users are at a loss.
        \end{itemize}
    \end{itemize}
  \end{block}
\end{frame}

\begin{frame}
  \begin{block}{Centralized or decentralized controls}
    \begin{itemize}
      \item Within the domain of a security policy, the same controls should be 
        enforced everywhere.

      \item Having a centralized entity to do this makes it easy to achieve 
        uniformity, however, this entity may become a bottleneck.

      \item A distributed solution might be more efficient, however, then you 
        must ensure they all enforce consistently with each other.
    \end{itemize}
  \end{block}
\end{frame}

\subsection{The layer below}

\begin{frame}
  \begin{itemize}
    \item Every security mechanism defines a \emph{security perimeter}.

    \item The parts of a system which can malfunction without breaking the 
      mechanism are said to be outside the perimeter.

    \item The parts of the system that can disable the mechanism are within the 
      perimeter.
  \end{itemize}
\end{frame}

\begin{frame}
  \begin{itemize}
    \item Attackers will try to bypass security mechanisms.

    \item How do you ensure an attacker cannot get access to the layer below 
      the security mechanism?
  \end{itemize}
\end{frame}

\begin{frame}
  \begin{itemize}
    \item Recovery tools, read the sectors directly from the disk; logical 
      access control is implemented in the operating system.

    \item Buffer overruns, a value assigned to a variable is too large for the 
      memory buffer allocated; memory allocated for other variables may be 
      overwritten.

    \item Side-channel analysis, look at the time different operations take to 
      perform, look at power consumption.

    \item JavaScript to perform security checks?
      The client can use a Web page without JavaScript enabled.

  \end{itemize}
\end{frame}


%%%%%%%%%%%%%%%%%%%%%%

\begin{frame}[allowframebreaks]
  \small
  \printbibliography{}
\end{frame}

