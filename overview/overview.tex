% $Id$
%\documentclass[handout]{beamer}
\documentclass{beamer}
\usepackage[utf8]{inputenc}
\usepackage[T1]{fontenc}
\usepackage[swedish,british]{babel}
\usepackage{url}
\usepackage{graphicx}
\usepackage{color}
\usepackage{keystroke}

\setbeamertemplate{bibliography item}[text]
\usepackage[natbib,style=alphabetic,maxbibnames=99]{biblatex}
\addbibresource{overview.bib}

\mode<presentation>{%
  \usetheme{Berlin}
  \setbeamercovered{transparent}
  %\usecolortheme{seagull}
}
\setbeamertemplate{footline}%{miniframes theme}
{%
  \begin{beamercolorbox}[colsep=1.5pt]{upper separation line foot}
  \end{beamercolorbox}
  \begin{beamercolorbox}[ht=2.5ex,dp=1.125ex,%
    leftskip=.3cm,rightskip=.3cm plus1fil]{author in head/foot}%
    \leavevmode{\usebeamerfont{author in head/foot}\insertshortauthor}%
    \hfill%
    {\usebeamerfont{institute in head/foot}\usebeamercolor[fg]{institute in head/foot}\insertshortinstitute}%
  \end{beamercolorbox}%
  \begin{beamercolorbox}[ht=2.5ex,dp=1.125ex,%
    leftskip=.3cm,rightskip=.3cm plus1fil]{title in head/foot}%
    {\usebeamerfont{title in head/foot}\insertshorttitle} \hfill     \insertframenumber%
  \end{beamercolorbox}%
  \begin{beamercolorbox}[colsep=1.5pt]{lower separation line foot}
  \end{beamercolorbox}
}

\title[Intro to Info Sec]{%
  Introduction to Information Security
}
\author{Daniel Bosk}
\institute[MIUN ICS]{%
  Department of Information and Communication Systems,\\
  Mid Sweden University, SE-851\,70 Sundsvall.
}
\date{\today}

% XXX the underscores in the logotype filename breaks compilation
% XXX for some weird reason
%\pgfdeclareimage[height=0.65cm]{university-logo}{MU_logotyp_int_CMYK.pdf}
%\logo{\pgfuseimage{university-logo}}

\AtBeginSection[]{%
  \begin{frame}<beamer>
		\tableofcontents[currentsection]
	\end{frame}
}

\begin{document}

\begin{frame}
  \titlepage{}
\end{frame}

\begin{frame}
	\tableofcontents
	% You might wish to add the option [pausesections]
\end{frame}


% Since this a solution template for a generic talk, very little can
% be said about how it should be structured. However, the talk length
% of between 15min and 45min and the theme suggest that you stick to
% the following rules:  

% - Exactly two or three sections (other than the summary).
% - At *most* three subsections per section.
% - Talk about 30s to 2min per frame. So there should be between about
%   15 and 30 frames, all told.


\section{What's this about?}

\subsection{Security Strategies}

\begin{frame}
  \begin{itemize}
    \item The main purpose of security is to protect assets.

    \item In our \emph{(security) policy} we define what is to be accomplished, 
      the goals, e.g.~who may access what asset and how.

    \item We then have \emph{mechanisms} to help us enforce our policy, 
      e.g.~cryptography.
  \end{itemize}
\end{frame}

\begin{frame}
  \begin{itemize}
    \item Each of our mechanisms we can say is more or less \emph{trustworthy}:
      A trustworthy mechanism will not break our security policy.

    \item The aim of this course is to give you an idea of how to determine 
      what is a trustworthy mechanism.
  \end{itemize}
\end{frame}

\begin{frame}
  \begin{itemize}
    \item Our protection strategies can be divided into the following:
      \begin{itemize}
        \item Prevention, taking measures that prevent your assets from being 
          damaged.

        \item Detection, taking measures that allow detection of when, how, and 
          by who an asset has been damaged.

        \item Reaction, taking measures that allow to recover assets or recover 
          from damage to assets.
      \end{itemize}
  \end{itemize}
\end{frame}

\begin{frame}
  \begin{example}[Private property]
    \begin{description}
      \item[Prevention] Locks on doors, window bars, surrounding walls, \dots
      \item[Detection] Stolen items are missing, burglar alarms, video 
        surveillance, \dots
      \item[Reaction] Call the police, replace stolen items (insurance?), \dots
    \end{description}
  \end{example}

  \begin{example}[E-commerce]
    \begin{description}
      \item[Prevention] Encrypt orders, rely on merchants checking identities, 
        \dots
      \item[Detection] An unauthorised transaction appears on your bank 
        statement, \dots
      \item[Reaction] Complain to bank, ask for new card, \dots
    \end{description}
  \end{example}
\end{frame}

\subsection{Data and Information}

\begin{frame}
  \begin{definition}[Data and information~\cite{hansen1973operating}]
    [Data is the] Physical phenomena chosen by convention to represent certain 
    aspects of of our conceptual and real world.
    \emph{The meanings we assign to data are called information.} [my emphasis]
    Data is used to transmit and store information and to derive new 
    information by manipulating the data according to formal rules.
  \end{definition}
\end{frame}

\begin{frame}
  \begin{example}
    \begin{itemize}
      \item \(x + 1 = 0\iff x = -1\): the two equations are data, we use the 
        formal rules of mathematics to derive the value of \(x\).

      \item Timestamps of TCP packets with port set to 80 or 443, formal rules 
        of statistical analysis.
    \end{itemize}
  \end{example}
\end{frame}

\subsection{Security Objectives}

\begin{frame}
  \begin{description}
    \item[Confidentiality] Concerns unauthorised disclosure of information.
    \item[Integrity] Concerns unauthorised modification.
    \item[Availability] Concerns unauthorised withholding of information or 
      resources.

      \pause{}

    \item[Authenticity] Concerns the identity of principals in systems.
    \item[Accountability (non-repudiation)] Concerns proof that some principal 
      was involved in some event.
    \item[\dots]
  \end{description}
\end{frame}

\begin{frame}{Confidentiality}
  \begin{itemize}
    \item Prevent unauthorised \emph{reading}.

    \item Think about what to hide: the content of a document, or the 
      document's existence?
  \end{itemize}
\end{frame}

\begin{frame}{Privacy}
  \begin{itemize}
    \item Privacy is different, this concerns protection of personal 
      information.

    \item The users should be in control of their data and of the information 
      about their activities.

    \item Varying definitions: also the right to be left alone.
  \end{itemize}
\end{frame}

\begin{frame}{Integrity}
  \begin{itemize}
    \item Prevent unauthorised \emph{writing}.

    \item Data integrity: \enquote{The state that exists when computerized data 
        is the same as that in the source document and has not been exposed to 
        accidental or malicious alteration or destruction.}

    \item Concerns detection and correction of intentional and unintentional 
      modifications of data.
  \end{itemize}
\end{frame}

\begin{frame}{Integrity}
  \begin{itemize}
    \item Clark and Wilson:
      ``No user of the system, even if authorized, may be permitted to modify 
      data in such a way that assets or accounting records of the company are 
      lost or corrupted.''

    \item I.e.~make sure that everything is as it is supposed to.

    \item Integrity is a prerequisite to many other security services.
  \end{itemize}
\end{frame}

\begin{frame}{Availability}
  \begin{itemize}
    \item This is the property of being available and usable upon demand by an 
      authorized principal.

    \item Denial of Service (DoS) is an attack on availability which prevents 
      authorized access to resources or the delaying of time-critical 
      operations.

    \item A very important part of security, unfortunately not many methods for 
      accomplishing this are available.

    \item Distributed Denial of Service (DDoS) gets much attention, this can 
      also be seen as a reliability problem (unintentional).
  \end{itemize}
\end{frame}

\begin{frame}{Availability}
  \begin{example}[Smurf attack]
    \begin{itemize}
      \item Attacker sends ICMP echo request to a broadcast address with the 
        victim's address as the spoofed sender address.

      \item The echo request is distributed to all nodes in the broadcast 
        range.

      \item All nodes replies to the echo request, and the replies are sent to 
        the victim -- the victim is flooded.

      \item Depending on the size of the broadcast range, there is 
        a considerable amplification.
    \end{itemize}
  \end{example}
\end{frame}

\begin{frame}
  \begin{example}[DDoS]
    \begin{itemize}
      \item Attackers infect \(x\cdot 10^5\) devices (smartphones, IoT-stuff) 
        with malware.
      \item Attackers commands all devices to send requests to a given address.
      \item No one can communicate with that address under that load.
    \end{itemize}
  \end{example}
\end{frame}

\begin{frame}{Non-Repudiation}
  \begin{itemize}
    \item \emph{Non-repudiation} concerns unforgeable evidence that a specific 
      action has occurred.

    \item \emph{Non-repudiation of origin}: protects against a sender of data 
      denying data was sent.

    \item \emph{Non-repudiation of delivery}: protects against a receiver 
      denying data was received.

    \item Note: what is meant my received?
      E.g.~a mail delivered to your mailbox.

    \item Also note that a simple audit log doesn't necessarily give 
      non-repudiation, it might have been forged by the system administrator.

    \item It's usually accomplished by using crypto.
  \end{itemize}
\end{frame}

\begin{frame}{Non-Repudiation}
  \begin{itemize}
    \item A commonly found definition: ``Non-repudiation provides irrefutable 
      evidence about some event.''

    \item Is anything ever irrefutable?

    \item Non-repudiation generates mathematical evidence.

    \item This does not necessarily mean it is accepted, e.g.~in court of law.

    \item In Sweden, this is regulated in law SFS 2000:832 (among others).
  \end{itemize}
\end{frame}

\subsection{Security and Reliability}

\begin{frame}
  \begin{block}{Reliability}
    \begin{itemize}
      \item Reliability addresses the consequences of unintentional errors.

      \item On a PC (offline), you are in control of the software components 
        sending input to each other.

      \item The aim is to avoid mistakes.
    \end{itemize}
  \end{block}

  \pause{}

  \begin{block}{Security}
    \begin{itemize}
      \item Once online, hostile adversaries can provide input.

      \item Protection against mistakes is not enough --- they will ensure to 
        make them for you.

      \item And they will do things that you can never accomplish by mistake.
    \end{itemize}
  \end{block}
\end{frame}

\begin{frame}
  \begin{itemize}
    \item To make software more reliable, it is tested against typical usage 
      patterns.

    \item To make software more secure, it has to be tested against non-typical 
      usage patterns.

    \item In fact, you can never be sure the software is secure by just testing 
      -- you need to prove it secure.
  \end{itemize}
\end{frame}


\section{Dilemmas of Security}

\subsection{The Fundamental Dilemma of Security}

\begin{frame}
  \begin{block}{Situation}
    Security-unaware users have specific security requirements but no security 
    expertise.
  \end{block}

  \begin{block}{Dilemma}
    \begin{itemize}
      \item If you provide them with a standard (\enquote{best-practice}) 
        solution it might not meet their requirements.

      \item If you want to tailor your solution to the users' needs, they may 
        be unable to tell you what they require.

    \end{itemize}
  \end{block}
\end{frame}

\subsection{Another Dilemma of Security}

\begin{frame}
  \begin{itemize}
    \item The other dilemma is the conflict between security and 
      usability.
      
    \item Security mechanisms may need additional computational resources.

    \item Security interferes with the ordinary working pattern which users are 
      accustomed to.

    \item Effort has to be put into managing security --- or we do security 
      correct and aligned with usability!
  \end{itemize}
\end{frame}


\section{Principles of Security}

\subsection{Fundamental Design Decisions}

\begin{frame}
  \begin{enumerate}
    \item Where to focus security controls?
    \item Where to place security controls?
    \item Complexity or assurance?
    \item Centralised or decentralised control?
    \item Blocking access to the layer below?
  \end{enumerate}
\end{frame}

\subsection{Focus and Placement of Control}

\begin{frame}
  \begin{itemize}
    \item Focus of control may be on data, operations, or users.

    \item If we look at the control of integrity, its requirements may refer to 
      rules on:
      \begin{itemize}
        \item Format and content of data items, e.g.~account balance must be 
          integer.

        \item Operations that may be performed on a data item, e.g.~credit, 
          debit and transfer.

        \item Users who are allowed access to a data item, e.g.~account holder 
          and bank clerk.
      \end{itemize}
  \end{itemize}
\end{frame}

%\begin{frame}
%  \begin{block}{Man-Machine Scale}
%    \begin{itemize}
%      \item Applications (focus on users, information)
%      \item Services (middleware)
%      \item Operating system
%      \item Operating system kernel
%      \item Hardware (focus on data, generic)
%    \end{itemize}
%  \end{block}
%\end{frame}


\subsection{Complexity or Assurance}

\begin{frame}
  \begin{itemize}
    \item Often the location of a security mechanism in the man--machine scale 
      correlates with its complexity.

    \item Generic mechanisms are simple, applications are usually feature rich.

    \item Back to the fundamental dilemma:
      \begin{itemize}
        \item Simple generic mechanisms may not match specific security 
          requirements.

        \item To choose the right features from a rich selection, you need to 
          be a security expert.

        \item Security-unaware users are at a loss.
      \end{itemize}
  \end{itemize}
\end{frame}

\subsection{Centralized or Decentralized Controls}

\begin{frame}
  \begin{itemize}
    \item Within the domain of a security policy, the same controls should be 
      enforced everywhere.

    \item Having a centralized entity to do this makes it easy to achieve 
      uniformity, however, this entity may become a bottleneck.

    \item A distributed solution might be more efficient, however, then you 
      must ensure they all enforce consistently with each other.
  \end{itemize}
\end{frame}

\subsection{The Layer Below}

\begin{frame}
  \begin{itemize}
    \item Every security mechanism defines a \emph{security perimeter}.

    \item The parts of a system which can malfunction without breaking the 
      mechanism are said to be outside the perimeter.

    \item The parts of the system that can disable the mechanism are within the 
      perimeter.
  \end{itemize}
\end{frame}

\begin{frame}
  \begin{itemize}
    \item Attackers will try to bypass security mechanisms.

    \item How do you ensure an attacker cannot get access to the layer below 
      the security mechanism?
  \end{itemize}
\end{frame}

\begin{frame}
  \begin{itemize}
    \item Recovery tools, read the sectors directly from the disk; logical 
      access control is implemented in the operating system.

    \item Buffer overruns, a value assigned to a variable is too large for the 
      memory buffer allocated; memory allocated for other variables may be 
      overwritten.

    \item Side-channel analysis, look at the time different operations take to 
      perform, look at power consumption.

    \item JavaScript to perform security checks?
      The client can use a Web page without JavaScript enabled.

  \end{itemize}
\end{frame}


%%%%%%%%%%%%%%%%%%%%%%

\begin{frame}[allowframebreaks]
	\small
  \printbibliography{}
\end{frame}

\end{document}

