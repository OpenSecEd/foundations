In this learning session we will cover the foundations of security.
By this we mean what security is all about, \eg what properties we are 
interested in and what we want to achieve in our security work.

We will focus on Gollmann's chapter on \enquote{Foundations of Computer 
Security}~\cite[Chap.\ 3]{Gollmann2011cs}.
There he attempts at a definition of Computer Security and related terms, \eg 
confidentiality, integrity, and availability, which we need for our treatment of 
the topic.
After reading this chapter you you are encouraged to do exercises 3.2, 3.5, 
3.6, 3.7 and 3.8 in~\cite{Gollmann2011cs}.
Anderson also covers this in Chapter 1 of~\cite{Anderson2008sea}.
He also treats a wider area than just \emph{computer} security, which is good 
for us, he covers many aspects of security in different examples.

Finally, you should read 
\citetitle{HowToDesignSecurityExperiments}~\cite{HowToDesignSecurityExperiments}.
This paper discusses the scientific method of the security field.
For a more in-depth reflection on the state of security as a scientific pursuit, 
we recommend \citetitle{SecurityAsAScience}~\cite{SecurityAsAScience}.
