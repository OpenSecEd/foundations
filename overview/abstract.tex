In this learning session we will outline the foundations of security.
By this we mean what security is all about, e.g.\ what properties we are 
interested in and what we want to achieve in our security work.
We will also discuss how we can reason about security, i.e.\ the scientific 
method, how we can falsify claims about security.
We need this to reflect about what we learn and how we can verify new 
knowledge.

More specifically, after this session you should be able to
\begin{itemize}
  \item \emph{understand} the different types of models used in the security 
    field and \emph{compare} their different advantages and limitations.
  \item \emph{critically reflect} on claims about security and what is needed 
    to evaluate them.
\end{itemize}

The reading material consists of Gollmann's chapter on \enquote{Foundations of 
  Computer Security}~\cite[Chap.\ 3]{Gollmann2011cs}.
There he attempts at a definition of Computer Security and related terms, e.g.\ 
confidentiality, integrity and availability, which we need for our treatment of 
the topic.
After reading this chapter you you are encouraged to do exercises 3.2, 3.5, 
3.6, 3.7 and 3.8 in~\cite{Gollmann2011cs}.

Anderson also covers this in Chapter 1 of~\cite{Anderson2008sea}, \enquote{What 
  is security engineering?}.
He also treats a wider area than just \emph{computer} security, which is good 
for us, he covers many aspects of security in different examples.

Finally, the scientific discussion is treated in 
\citetitle{ComputerSecurityExperiments}~\cite{ComputerSecurityExperiments} by 
\citeauthor{ComputerSecurityExperiments}.
